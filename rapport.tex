% rapport.tex (corrigé)
\documentclass[12pt,a4paper]{report}
\usepackage[T1]{fontenc}
\usepackage[utf8]{inputenc}
\usepackage[french]{babel}
\usepackage{csquotes}
\usepackage{geometry}
\geometry{margin=2.5cm}
\usepackage{graphicx}
\usepackage{caption}
\usepackage{subcaption}
\usepackage{hyperref}
\usepackage{float}
\usepackage{booktabs}
\usepackage{enumitem}
\usepackage{fancyhdr}
\usepackage{xcolor}
\usepackage{listings}
\usepackage{titling}
\usepackage{setspace}
\usepackage{ifthen} % pour tests conditionnels de fichiers

% Header / Footer
\pagestyle{fancy}
\fancyhf{}
\lhead{Design Patterns — Leçon 5}
\rhead{Donfack Pascal — M2-GI}
\cfoot{\thepage}

% Hyperref setup
\hypersetup{
  colorlinks=true,
  linkcolor=blue,
  citecolor=blue,
  urlcolor=cyan,
  pdftitle={Rapport — Patrons structurels (Leçon 5)},
  pdfauthor={Donfack Pascal}
}

% Style listings (dark theme)
\definecolor{bg}{RGB}{40,44,52}
\definecolor{fg}{RGB}{220,220,220}
\definecolor{keyword}{RGB}{86,156,214}
\definecolor{string}{RGB}{206,145,120}
\definecolor{comment}{RGB}{106,153,85}
\lstdefinestyle{dark}{
  backgroundcolor=\color{bg},
  basicstyle=\ttfamily\footnotesize\color{fg},
  keywordstyle=\color{keyword}\bfseries,
  stringstyle=\color{string},
  commentstyle=\color{comment}\itshape,
  numberstyle=\tiny\color{fg},
  frame=single,
  rulecolor=\color{black!60},
  xleftmargin=2pt,
  xrightmargin=2pt,
  breaklines=true,
  numbers=left,
  numbersep=6pt,
  showstringspaces=false
}
\lstset{style=dark, language=Python}

% Title file
% title.tex
\begin{center}
  \vspace*{2cm}
  {\LARGE \textbf{École Nationale Supérieure Polytechnique de Yaoundé} \par}
  \vspace{1.5cm}
  {\Huge \textbf{Rapport de Travaux Pratiques}\\[0.5em]
  \textbf{Leçon 5 -- Patrons structurels}} \par
  \vspace{1.5cm}
  {\Large \textbf{Sujet :} Implémentation du Patron Composite (Menu de restaurant) et Iterator} \par
  \vspace{2cm}
  \begin{tabular}{rl}
    \textbf{Étudiant :} & Donfack Pascal \\
    \textbf{Classe :} & M2-GI \\
    \textbf{Encadreur :} & Pr Kouamou \\
    \textbf{Établissement :} & École Nationale Supérieure Polytechnique de Yaoundé \\
  \end{tabular}
  \vfill
  {\large \today \par}
\end{center}


\begin{document}
\onehalfspacing
\maketitle
\pagenumbering{roman}
\tableofcontents
\listoffigures
\lstlistoflistings
\clearpage
\pagenumbering{arabic}

% -------------------------
\chapter*{Résumé}
\addcontentsline{toc}{chapter}{Résumé}
Ce rapport présente les solutions aux exercices de la \textbf{Leçon 5 -- Structural Patterns} \cite{lesson5} :
implémentation en Python du \textbf{Composite} (arbre du menu), implémentation d'un itérateur récursif
(concept Iterator) et exemple \textbf{Head First} (Decorator — Starbuzz Coffee) \cite{headfirst2004}.
Le dépôt contient les sources, diagrammes et tests. Le code important est inclus directement dans ce rapport via \texttt{\textbackslash lstinputlisting} (extraits) ; les fichiers complets sont référencés par des liens GitHub pour consultation rapide \cite{repo}.

% -------------------------
\chapter{Introduction et contexte}
\label{ch:intro}
Les patterns de conception (Design Patterns) sont des solutions réutilisables à des problèmes récurrents en conception orientée objet (GoF) \cite{gof1995}. Ce travail se concentre sur les \emph{patrons structurels} (Composite, Iterator, Decorator) illustrés dans la Leçon 5 \cite{lesson5}.
Les exercices demandent :
\begin{enumerate}
  \item Reproduire en Python l'arbre du menu (Composite) et implémenter, dans la classe représentant le serveur (\emph{Waitress}), une méthode permettant d'afficher un menu spécifique.
  \item Produire le schéma général du patron \emph{Iterator}, préciser les participants et appliquer le schéma à l'exemple du menu.
  \item Explorer la démarche \emph{Head First} : produire un schéma et un exemple de code (Decorator).
\end{enumerate}

% -------------------------
\chapter{Patron Composite — Arbre du menu}
\label{ch:composite}

\section{Problème et intention}
Le problème est de représenter une hiérarchie partie-tout (menus et items) où le client peut traiter uniformément les feuilles (plats) et les composites (sous-menus). L'intention du patron Composite est de composer des objets en structures arborescentes et de permettre aux clients d'interagir de façon uniforme \cite{gof1995}.

\section{Modèle conceptuel appliqué au menu}
La structure implémentée contient :
\begin{itemize}
  \item \textbf{MenuComponent} : interface abstraite (API commune). (voir \href{https://github.com/Tiger-Foxx/design-patterns-impl/blob/main/src/menu_composite/menu_component.py}{\texttt{src/menu_composite/menu_component.py}})
  \item \textbf{MenuItem} : feuille (plat). (voir \href{https://github.com/Tiger-Foxx/design-patterns-impl/blob/main/src/menu_composite/menu_item.py}{\texttt{src/menu_composite/menu_item.py}})
  \item \textbf{Menu} : composite, contenant des enfants. (voir \href{https://github.com/Tiger-Foxx/design-patterns-impl/blob/main/src/menu_composite/menu.py}{\texttt{src/menu_composite/menu.py}})
  \item \textbf{Waitress} : le serveur (client) qui utilise l'API pour afficher, filtrer, et rechercher. (voir \href{https://github.com/Tiger-Foxx/design-patterns-impl/blob/main/src/menu_composite/waitress.py}{\texttt{src/menu_composite/waitress.py}})
\end{itemize}

\begin{figure}[H]
  \centering
  \includegraphics{docs/diagrams/menu_example.png}
  \caption{Schéma de l'arbre du menu (Composite)}
  \label{fig:menu_tree}
\end{figure}

\section{Code principal (extraits)}
\subsection{API Commune : \texttt{menu_component.py}}
\lstinputlisting[caption={menu\_component.py — interface commune}, label={lst:menu_component}]{src/menu_composite/menu_component.py}

\subsection{Feuille : \texttt{menu_item.py}}
\lstinputlisting[caption={menu\_item.py — MenuItem (feuille)}, label={lst:menu_item}]{src/menu_composite/menu_item.py}

\subsection{Composite : \texttt{menu.py}}
\lstinputlisting[caption={menu.py — Menu (composite)}, label={lst:menu}]{src/menu_composite/menu.py}

\subsection{Serveur (Waitress) — méthode \texttt{print\_specific\_menu}}
La classe \texttt{Waitress} implémente la fonctionnalité demandée : afficher un sous-arbre ou un item précis à partir du nom.
\lstinputlisting[caption={waitress.py — Serveur (méthodes d'affichage, recherche et print\_specific\_menu)}, label={lst:waitress}]{src/menu_composite/waitress.py}

\section{Discussion et alternatives}
Le choix d'utiliser une interface commune simplifie le code client. Pour de très grands arbres, considérer une sérialisation/lazy-loading ou l'utilisation d'une base (ex. PostgreSQL/PostGIS) si les menus sont volumineux ou dynamiques.

% -------------------------
\chapter{Patron Iterator — théorie et application au menu}
\label{ch:iterator}

\section{Participants (définition chirurgicale)}
Conformément au patron :
\begin{description}
  \item[Iterator :] définit l'interface d'accès séquentiel aux éléments (en Python : \texttt{\_\_iter\_\_} / \texttt{\_\_next\_\_}).
  \item[ConcreteIterator :] implémente l'itérateur (ex : \texttt{CompositeIterator}).
  \item[Aggregate :] interface qui crée l'itérateur (\texttt{create\_iterator()}).
  \item[ConcreteAggregate :] la collection concrète (\texttt{Menu}).
  \item[Client :] utilise l'itérateur (\texttt{Waitress}).
\end{description}

\begin{figure}[H]
  \centering
  \includegraphics{docs/diagrams/Schema-genral-iterator.png}
  \caption{Schéma général du pattern Iterator}
  \label{fig:iterator_general}
\end{figure}

\section{Implémentation — \texttt{composite\_iterator.py}}
Le fichier suivant implémente un itérateur en profondeur (DFS) qui permet de parcourir tout l'arbre sans exposer sa structure interne.
\lstinputlisting[caption={composite\_iterator.py — CompositeIterator (DFS)}, label={lst:composite_iterator}]{src/menu_composite/composite_iterator.py}

\section{Usage dans la Waitress}
La Waitress utilise \texttt{CompositeIterator} pour lister tous les éléments, trouver un sous-menu par nom et lister les plats végétariens.
Voir le code complet : \href{https://github.com/Tiger-Foxx/design-patterns-impl/blob/main/src/menu_composite/composite_iterator.py}{\texttt{composite\_iterator.py}}.

% -------------------------
\chapter{Head First Design Patterns — approche et exemple (Decorator)}
\label{ch:headfirst}

\section{Approche Head First}
La série \emph{Head First} privilégie l'apprentissage visuel et par l'exemple. L'exemple classique du livre pour le pattern \emph{Decorator} est \emph{Starbuzz Coffee} — un objet boisson décoré dynamiquement par des condiments \cite{headfirst2004}.

\section{Schéma}
\begin{figure}[H]
  \centering
  % affiche l'image seulement si elle existe dans le dépôt
  \IfFileExists{docs/diagrams/schema_headfirst_decorator.png}{
    \includegraphics[width=0.7\textwidth]{docs/diagrams/schema_headfirst_decorator.png}
  }{
    \fbox{\parbox[c][4cm][c]{0.7\textwidth}{\centering Image \texttt{schema\_headfirst\_decorator.png} manquante. Ajoutez-la dans docs/diagrams/ pour l'afficher.}}
  }
  \caption{Schéma conceptuel — Decorator (Head First)}
  \label{fig:decorator}
\end{figure}

\section{Exemple (Starbuzz Coffee) — code}
\lstinputlisting[caption={beverage.py — Classe de base (Head First)}, label={lst:beverage}]{src/headfirst_decorator/beverage.py}

\lstinputlisting[caption={condiments.py — Condiment Decorators (Head First)}, label={lst:condiments}]{src/headfirst_decorator/condiments.py}

Exécution de démonstration : \href{https://github.com/Tiger-Foxx/design-patterns-impl/blob/main/src/headfirst_decorator/starbuzz_demo.py}{\texttt{starbuzz\_demo.py}}.

% -------------------------
\chapter{Résultats, tests et mode d'emploi}
\label{ch:tests}
\section{Exécution du demo}
Le script de démonstration construit un menu d'exemple et exécute les scénarios demandés :
\begin{itemize}
  \item Affichage complet de la carte.
  \item Liste des éléments végétariens.
  \item Affichage d'un menu spécifique (ex : « Diner Menu »).
  \item Affichage d'un item spécifique (ex : « Apple Pie »).
\end{itemize}

\subsection*{Commande}
\begin{verbatim}
# depuis la racine du repo
python src/menu_composite/demo.py
\end{verbatim}

\subsection*{Tests unitaires}
\begin{verbatim}
# installer pytest si nécessaire
pip install pytest
pytest -q
\end{verbatim}
Le jeu de tests se trouve dans \href{https://github.com/Tiger-Foxx/design-patterns-impl/blob/main/tests/test_menu.py}{\texttt{tests/test\_menu.py}}.

\section{Liens vers le code complet (GitHub)}
Les fichiers complets (pour consultation ou revue rapide) sont disponibles à ces URLs :
\begin{itemize}
  \item \href{https://github.com/Tiger-Foxx/design-patterns-impl/blob/main/src/menu_composite/menu.py}{\texttt{menu.py}}
  \item \href{https://github.com/Tiger-Foxx/design-patterns-impl/blob/main/src/menu_composite/menu_item.py}{\texttt{menu_item.py}}
  \item \href{https://github.com/Tiger-Foxx/design-patterns-impl/blob/main/src/menu_composite/menu_component.py}{\texttt{menu_component.py}}
  \item \href{https://github.com/Tiger-Foxx/design-patterns-impl/blob/main/src/menu_composite/composite_iterator.py}{\texttt{composite_iterator.py}}
  \item \href{https://github.com/Tiger-Foxx/design-patterns-impl/blob/main/src/menu_composite/waitress.py}{\texttt{waitress.py}}
  \item \href{https://github.com/Tiger-Foxx/design-patterns-impl/tree/main/src/headfirst_decorator}{\texttt{headfirst\_decorator/}}
\end{itemize}

% -------------------------
\chapter{Conclusion}
Le travail satisfait les exigences pédagogiques : 
\begin{itemize}
  \item Reproduction de l'arbre du menu (Composite) en Python \cite{lesson5}.
  \item Implémentation d'un itérateur récursif (Iterator) adapté à la structure.
  \item Méthode du serveur (Waitress) pour afficher un menu spécifique.
  \item Exemple Head First (Decorator) avec code d'illustration.
\end{itemize}
Le dépôt contient les codes, diagrammes et tests ; le rapport intègre les extraits essentiels et référence les sources complètes sur GitHub \cite{repo,gof1995,headfirst2004}.

% -------------------------
\clearpage
\appendix
\chapter{Annexes — Code complet (sélection)}
% On inclut ici quelques listings supplémentaires si besoin
\lstinputlisting[caption={demo.py — script de démonstration}, label={lst:demo}]{src/menu_composite/demo.py}

% -------------------------
\clearpage
\addcontentsline{toc}{chapter}{Bibliographie}
\nocite{*} % force l'affichage de toutes les entrées de references.bib
\bibliographystyle{plain}
\bibliography{references}

\end{document}
